%        File: main.tex
%     Created: Mon Jan 30 10:00 AM 2017 I
% Last Change: Mon Jan 30 10:00 AM 2017 I
%
\documentclass[a4paper]{article}
\usepackage[]{palatino,hyperref}
\usepackage[backend=bibtex,style=numeric]{biblatex}
\usepackage{csquotes}
\addbibresource{sources.bib}
\title{Errors, Fakes, Destroyed Evidence and Historical Events}
\author{Jerin Philip}
\date{201401071}
\begin{document}
\maketitle

% What role do fakes, deliberate errors and forgotten/destroyed evidence
% play in recreating historical events?
\section{Introduction}
Consider a student taking notes in a classroom. With resources - words
from the teacher, writings in class and textbooks in hand, the student
manages to capture only a fragment of the content through his
perspective while making notes. A third party reading the notes cannot
completely deny there are truths in contained in the notebook, neither
will anyone be able to assert that the contents are fully correct if
inconsistencies are found in the original material and the notes.

A historian is no more than a well-trained but less naive student, but
the same errors still creep in. A work of history is an interpretation
of the events of the past through the perspective of the historian.  Not
all records of events are faithful to how it actually really went down.
History is written by the victors - the one of the losers are destroyed
or reused to glorify the victors further. Plenty of fake
evidences enter the records as time progresses.

What follows is an attempt to draw from a few historic anecdotes and
generalize that the fakes, deliberate errors and forgotten or destroyed
evidence plays a significant role in recreating historic events.
% Harry Potter - Snape's notebook.
% Quran - 72 Virgins
% Demonitisation.
% Hyderabad Operation Polo
% Colonial Story
% Nazi Story

\section{Cost of History}

Information theory, which studies quantification, storage, and
communication on information helps us make the following assertion - we
simply can't recreate historical events as a whole. A historian
recording events cannot simply quantify and store everything in a few
parchments or lines without loss of information. The loss is simply the
penalty we pay for having at least some information of the past. Hence
deliberate elimination of a few facts are a necessity, in scientific
terms. E.H Carr simply explains this \cite[p.9]{carr1961history} with an
anecdote - ``Caesar's crossing of that petty stream, the Rubicon, is a
fact of history, whereas the crossing of the Rubicon by millions of
other people before or since interests nobody at all.'' Hence a
historian has to make that selection for relevant information to
prevail.


\section{Breakdown of Relevance}

\subsection{Facts}

In most cultures' education systems, history has been categorised under
arts and not science. The student is taught that it's a collection of
facts - and most people in the process forget that it's a selective
arrangement of facts the way the historian liked it and hence there can
always be room for questions, counter arguments. The idea that history
consists of a compilation of maximum number of irrefutable and objective
facts is a heresy, argues E.H Carr\cite[p. 15]{carr1961history}.

\blockquote{The facts are available to the historian in documents,
    inscriptions and so on, like fish on the fish monger's slab. The
    historian collects them, takes them home, and cooks and serves them
    in whatever style appeals to him. \cite[p.  9]{carr1961history}}

In recreating a historical event, we do not look at the secondary
sources alone, or a contemporary historian's work. It is the
responsibility of the one who seeks knowledge to rely on ``auxilliary
sciences of history - archeology, epigraphy, numismatics, chronology and
so forth'', asserts Carr\cite[p.11]{carr1961history}.


\subsection{Errors}

In the domain of history, errors are also information. This is in stark
contrast to information theory, where errors are information corrupted
while sending over a channel with added noise or failure of delivery,
where it gives us simply nothing. But when a historian records an error,
it is less likely that it's a clerical error or a mistake - but one
packed with intent. When one try to corroborate a historian's writings
and the evidences to find inconsistencies, it gives us further room to
ask why the historian made the error.
% Bible, constantine.

Ancient India's lineage systems stands out in the epics -
\emph{Mahabharatha} and \emph{Ramayana}. We consider the Greek
counterparts \emph{Illiad} and \emph{Odyssey}, we find comparisons of
human beings to gods and drawing legitimacy from being a progeny of a
God. To the 21st century objective historian, it's child play to make
out the connection with the divine as absurd glorification of the
victors and the fallen to further glorify the victors. The big
\emph{but} here is although the text has huge errors - it lets a modern
day historian compare with real evidences and contemporary texts of
selective historical farts, as E.H Carr likes to call it\cite[p.
13]{carr1961history} to make sense of the then societal structure and
people's thinking.

\subsection{Forgotten Evidence}

The immediate question which arises of forgotten evidence is whether an
evidence recovered from the past missing from historical records is to
get into the mind of the historian. The factors which we get while
studying the historian helps in determining the likelihoods of the
omission being deliberate or irrelevance.

\subsection{Destroyed Evidence}



\section{Conclusion}

It is nearly impossible to compile all the facts without opinion mixed
in, hence we settle for a compromise. 

Mastering history implies mastering primary sources, the creators of
primary sources and recovered evidence and being able to question, think
and draw conclusions out of it.  
% Find and Cite EH Carr.
The process involves forming multiple hypotheses of explaining a
historical event and setting out find the likelihood of each being
correct or incorrect using whatever resources available.


\printbibliography 
\end{document}

