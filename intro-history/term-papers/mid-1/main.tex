%        File: main.tex
%     Created: Mon Jan 30 10:00 AM 2017 I
% Last Change: Mon Jan 30 10:00 AM 2017 I
%
\documentclass[a4paper]{article}
\usepackage[]{palatino,hyperref}
\usepackage[backend=bibtex,style=numeric]{biblatex}
\addbibresource{sources.bib}
\title{Errors, Fakes, Destroyed Evidence and Historical Events}
\author{Jerin Philip}
\date{201401071}
\begin{document}
\maketitle

% What role do fakes, deliberate errors and forgotten/destroyed evidence
% play in recreating historical events?
\section{Introduction}
Consider a student taking notes in a classroom. With resources - words
from the teacher, writings in class and textbooks in hand, the student
manages to capture only a fragment of the content through his
perspective while making notes. A third party reading the notes cannot
completely deny there are truths in what the notebook contains, neither
will anyone be able to assert that the contents are fully correct, if
they find inconsistencies in the original material and the notes.

A historian is a well-trained less naive student, but the same errors
still creep in. A work of history is an interpretation of the events of
the past through the eyes of the historian. 

% Harry Potter - Snape's notebook.
% Quran - 72 Virgins
% Demonitization.
% Hyderabad Operation Polo

\section{Trade-offs}
History: Facts or Opinions? 

\section{Conclusion}

It is nearly impossible to compile all the facts without opinion mixed
in, hence we settle for a compromise. 

\printbibliography 
\end{document}


